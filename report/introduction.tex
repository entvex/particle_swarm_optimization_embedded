\section{Introduction}

This report details the final project for the course "Embedded Real Time Systems". This introduction will inform the reader how to read the document. \\

Throughout the project, code snippets will be used to explain in detail, how they function and what certain code will do. In Listing \ref{lst:listingExample}, an example of such code snippet is shown.

\begin{lstlisting}[style=customc++, label={lst:listingExample}, caption={Example listing.}]
#include <stdio.h>
#define N 10
/* Block
* comment */

int main()
{
	int i;

	// Line comment.
	puts("Hello world!");

	for (i = 0; i < N; i++) {
		puts("LaTeX is also great for programmers!");
	}

	return 0;
}
\end{lstlisting}

This project follows the guide on what to include in the final project\cite{Bjerge2017}. This short document will be referred to often, and a mainstream of referring the document is used. In the start of most chapters a description will follow. The description is a snippet of the final project guide, in case the text is in a square box. Example follows.\\

Description:

\noindent\fbox{
	\parbox{\textwidth}{
		In this exercise, you should propose and demonstrate the use of a methodology based on SysML/UML and related profiles for the design of an Embedded Real Time project. The methodology must be applied for the design of a system that you define and specifies. The project should include a SysML/UML model with architectures for alternative SoPC solutions. A model should be developed to evaluate and verify the system in terms of hardware and software processing modules for a selected architecture. Finally part of the system should be implemented and verified on the ZYBO board. \cite{Bjerge2017} 
	}
}