\subsection{Behavior}\label{sc:behavior}
In SysML it is possible to model the behavior of the system using Activity Diagrams which describe the actions in a given Activity that controls the flow of input/output from one Activity to another. Sequence Diagram main focus is the communication between different objects in time. 

\subsubsection{Activity Diagram}\label{ssc:activitydiagram}
Activity diagram created to show the control flowing from one Activity to another and they are good at visualizing the logic such as of conditional structures, loops and concurrency. A way to look at it is to think about it as a complex flowchart.

\subsubsection{Sequence Diagram}\label{ssc:sequencediagram}
Sequence diagrams is used to model how parts of a block interact by the means of API calls signals or messages. This is useful to get an overview of how software works by using methods or functions between software components. But but be noted that Sequence Diagrams are not strictly for software.  